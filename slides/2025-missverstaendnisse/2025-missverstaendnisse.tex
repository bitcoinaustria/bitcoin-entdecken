\documentclass[aspectratio=169,t]{beamer}
\usepackage[utf8]{inputenc}
\usepackage{xcolor}
\usepackage{graphicx}
\usepackage{tikz}
\usepackage{amsmath}
\usepackage{amsfonts}
\usepackage{booktabs}

% Bitcoin orange color theme
\definecolor{bitcoinorange}{RGB}{247,147,26}
\definecolor{bitcoindark}{RGB}{33,33,33}
\definecolor{bitcoinlight}{RGB}{255,255,255}

% Custom theme
\usetheme{Madrid}
\usecolortheme{seahorse}

% Customize colors
\setbeamercolor{palette primary}{bg=bitcoinorange,fg=white}
\setbeamercolor{palette secondary}{bg=bitcoindark,fg=white}
\setbeamercolor{palette tertiary}{bg=bitcoinorange!70,fg=white}
\setbeamercolor{palette quaternary}{bg=bitcoinorange!50,fg=white}
\setbeamercolor{structure}{fg=bitcoinorange}
\setbeamercolor{frametitle}{bg=bitcoinorange,fg=white}
\setbeamercolor{title}{bg=bitcoindark,fg=white}
\setbeamercolor{block title}{bg=bitcoinorange,fg=white}
\setbeamercolor{block body}{bg=bitcoinorange!10}

% Title information
\title[Bitcoin Missverständnisse]{Bitcoin: Häufige Missverständnisse und was wirklich dahinter steckt}
\subtitle{Ein faktenbasierter Impulsvortrag}
\author{Präsentation}
\date{\today}
\institute{Bitcoin Entdecken / Bitcoin Austria}

% Custom footer with frame numbers only
\setbeamertemplate{footline}{
    \leavevmode%
    \hbox{%
        \begin{beamercolorbox}[wd=\paperwidth,ht=2.25ex,dp=1ex,right]{date in head/foot}%
            \insertframenumber{} / \inserttotalframenumber\hspace*{2ex}
        \end{beamercolorbox}%
    }
    \vskip0pt%
}

% Custom frametitle with logo on the right
\setbeamertemplate{frametitle}{
    \begin{beamercolorbox}[wd=\paperwidth,ht=2.5ex,dp=1ex]{frametitle}
        \hbox to \paperwidth{%
            \begin{beamercolorbox}[wd=0.75\paperwidth,left]{frametitle}%
                \hspace*{1ex}\usebeamerfont{frametitle}\insertframetitle%
            \end{beamercolorbox}%
            \begin{beamercolorbox}[wd=0.25\paperwidth,right]{frametitle}%
                \hfill\raisebox{-0.5ex}{\includegraphics[height=3ex,keepaspectratio]{logo-bitcoin-entdecken.pdf}}\hspace*{1ex}%
            \end{beamercolorbox}%
        }
    \end{beamercolorbox}
}

% Macro for misconception slides with consistent layout
\newcommand{\misconceptionslide}[3]{%
    % #1: Misconception text (in quotes)
    % #2: Facts list items
    % #3: Fazit text
    \begin{columns}[T]
        \begin{column}[T]{0.48\textwidth}
            \begin{block}{Das Missverständnis}
                \vspace{0.5em}
                \textit{#1}
                \vspace{0.5em}
            \end{block}
        \end{column}
        \hspace{0.02\textwidth}
        \begin{column}[T]{0.48\textwidth}
            \onslide<2->{
                \begin{block}{Die Fakten}
                    \vspace{0.3em}
                    \begin{itemize}
                        \setlength{\itemsep}{0.3em}
                        #2
                    \end{itemize}
                    \vspace{0.3em}
                \end{block}
            }
        \end{column}
    \end{columns}
    
    \vfill
    
    \onslide<2->{
        \textcolor{bitcoinorange}{\textbf{Fazit:}} #3
    }
}

\begin{document}

% Title slide
\begin{frame}
    \titlepage
    \begin{center}
        \textcolor{bitcoinorange}{\rule{0.8\textwidth}{2pt}}
    \end{center}
\end{frame}

% Table of contents
\begin{frame}{Überblick}
    \tableofcontents
\end{frame}

\section{Einleitung}

\begin{frame}{Warum entstehen Bitcoin-Missverständnisse?}
    \begin{itemize}
        \item \textbf{Komplexe Technologie} - Schwer verständlich für Laien
        \item \textbf{Medienverzerrung} - Sensationelle Berichterstattung
        \item \textbf{Schnelle Entwicklung} - Veraltete Informationen kursieren
        \item \textbf{Emotionale Diskussionen} - Fakten vs. Meinungen
        \item \textbf{Interessenskonflikte} - Verschiedene Akteure mit eigenen Agenden
    \end{itemize}
    \vspace{1em}
    \begin{alertblock}{Ziel dieser Präsentation}
        Fakten statt Mythen - Was sagt die aktuelle Forschung?
    \end{alertblock}
\end{frame}

\section{Die 9 häufigsten Missverständnisse}

\subsection{1. Umweltauswirkungen}

\begin{frame}<1-2>{Missverständnis 1: Umweltzerstörung}
    \misconceptionslide{"Bitcoin-Mining verbraucht zu viel Energie und zerstört die Umwelt"}{%
        \item Nur 0,54\% des globalen Stromverbrauchs
        \item 87\% der Mining-Hardware wird recycelt
        \item Zunehmend erneuerbare Energiequellen
        \item Weniger umweltschädlich als Goldbergbau
    }{Mining-Effizienz steigt kontinuierlich, Umweltimpact nimmt ab.}
\end{frame}

\subsection{2. Volatilität}

\begin{frame}<1-2>{Missverständnis 2: Zu volatil für praktische Nutzung}
    \misconceptionslide{"Bitcoin ist zu volatil, um als Währung oder Wertaufbewahrung zu funktionieren"}{%
        \item Volatilität nimmt langfristig ab
        \item Weniger volatil als 33 S\&P 500 Aktien
        \item Typisch für neue Asset-Klassen
        \item Institutionelle Adoption stabilisiert
    }{Reifung des Marktes führt zu weniger Volatilität.}
\end{frame}

\subsection{3. Regulierung}

\begin{frame}<1-2>{Missverständnis 3: Regulatorische Unsicherheit}
    \misconceptionslide{"Bitcoin hat keine rechtliche Grundlage und wird verboten werden"}{%
        \item Regulierungsrahmen entwickeln sich schnell
        \item USA und EU schaffen klare Gesetze
        \item Institutionelle Compliance wächst
        \item Verbote schwer durchsetzbar
    }{Rechtliche Klarheit nimmt weltweit zu.}
\end{frame}

\subsection{4. Skalierbarkeit}

\begin{frame}<1-2>{Missverständnis 4: Technische Limitierungen}
    \misconceptionslide{"Bitcoin kann nur 3-4 Transaktionen pro Sekunde verarbeiten"}{%
        \item Lightning Network: Millionen TPS möglich
        \item Bitcoin = Basis-Settlement-Layer
        \item Layer-2-Lösungen entwickeln sich
        \item Sicherheit wichtiger als reine Geschwindigkeit
    }{Skalierung durch Layer-2-Architektur gelöst.}
\end{frame}

\subsection{5. Kriminalität}

\begin{frame}<1-2>{Missverständnis 5: Kriminalitäts-Tool}
    \misconceptionslide{"Bitcoin wird hauptsächlich für illegale Aktivitäten verwendet"}{%
        \item Weniger als 1\% aller Transaktionen sind illegal (2024)
        \item Traditionelle Systeme haben mehr Kriminalität
        \item Blockchain ist transparent und nachverfolgbar
        \item Bargeld weiterhin bevorzugt für Illegales
    }{Bitcoin ist schlechter für Kriminalität als traditionelle Methoden.}
\end{frame}

\subsection{6. Spekulationsblase}

\begin{frame}<1-2>{Missverständnis 6: Reine Spekulation ohne Wert}
    \misconceptionslide{"Bitcoin ist eine Spekulationsblase ohne fundamentalen Wert"}{%
        \item Wertspeicher für langfristiges Sparen
        \item Netzwerkeffekte schaffen Mehrwert
        \item Institutionelle Adoption validiert Use-Case
        \item Knappheit (21 Mio. Limit) ähnlich wie Gold
    }{Fundamentaler Nutzen wächst über reine Spekulation hinaus.}
\end{frame}

\subsection{7. Praktische Adoption}

\begin{frame}<1-2>{Missverständnis 7: Keine praktische Nutzung}
    \misconceptionslide{"Bitcoin hat keine praktische Anwendung im Alltag"}{%
        \item Wachsende Händler-Akzeptanz
        \item Legales Zahlungsmittel (El Salvador, CAR)
        \item Corporate Treasury Asset
        \item Grenzüberschreitende Zahlungen
    }{Praktische Anwendungen wachsen kontinuierlich.}
\end{frame}

\subsection{8. Zentralbank-Bedenken}

\begin{frame}<1-2>{Missverständnis 8: Bedrohung für Geldpolitik}
    \misconceptionslide{"Bitcoin bedroht die Geldpolitik und wird von Regierungen verboten"}{%
        \item Ergänzt traditionelle Finanzsysteme
        \item Regierungen entwickeln crypto-freundliche Gesetze
        \item Zentralbanken erforschen digitale Währungen
        \item Hedge gegen Geldpolitik-Versagen
    }{Koexistenz statt Konkurrenzkampf mit traditioneller Finanzwelt.}
\end{frame}

\subsection{9. Weitere Mythen}

\begin{frame}<1-2>{Missverständnis 9: Technische Sicherheit}
    \misconceptionslide{"Bitcoin kann gehackt werden, hat keine Deckung und Mining ist sinnlose Verschwendung"}{%
        \item 99,98\% Uptime seit 2009
        \item Dezentralisierung = Unhackbarkeit
        \item Deckung durch Rechenarbeit \& Adoption
        \item Mining sichert das Netzwerk
    }{Technische Robustheit seit über 15 Jahren bewiesen.}
\end{frame}

\section{Zusammenfassung}

\begin{frame}{Was lernen wir daraus?}
    \begin{alertblock}{Schlüsselerkenntnisse}
        \begin{enumerate}
            \item Viele Kritikpunkte basieren auf \textbf{veralteten Informationen}
            \item Die Bitcoin-Technologie entwickelt sich \textbf{schnell weiter}
            \item \textbf{Faktendaten} widersprechen oft den Mediennarrativen
            \item \textbf{Institutionelle Adoption} validiert Bitcoins Legitimität
            \item \textbf{Umweltauswirkungen} nehmen ab, während Nutzen steigt
        \end{enumerate}
    \end{alertblock}
    
    \vspace{1em}
    \begin{block}{Empfehlung}
        \textcolor{bitcoinorange}{\textbf{Informieren Sie sich aus aktuellen, wissenschaftlichen Quellen!}}
    \end{block}
\end{frame}

\begin{frame}{Quellen und weitere Informationen}
    \footnotesize
    \begin{itemize}
        \item Cambridge Centre for Alternative Finance
        \item Chainalysis Crime Reports 2024
        \item Nature Scientific Reports
        \item Zentralbank-Publikationen (EZB, Fed, Bundesbank)
        \item Industrie-Reports großer Institutionen
        \item Peer-reviewed akademische Studien
    \end{itemize}
    
    \vspace{1em}
    \begin{center}
        \textcolor{bitcoinorange}{\Large Fragen und Diskussion?}
        \\[1em]
        \textcolor{bitcoinorange}{\rule{0.5\textwidth}{2pt}}
    \end{center}
\end{frame}

\end{document}
